\documentclass[10pt]{article}

%\usepackage{fancyhdr}
 
%\pagestyle{headings}
%\markright{John Smith}

\date{}

\usepackage{amsmath}    % need for subequations
\usepackage{graphicx}   % need for figures
\usepackage{verbatim}   % useful for program listings
\usepackage{color}      % use if color is used in text
\usepackage{subfigure}  % use for side-by-side figures
\usepackage[colorlinks=true,citecolor=blue]{hyperref}   % use for hypertext links
\usepackage{lipsum}
\usepackage{url}

\usepackage[margin=1in]{geometry}
\usepackage{lastpage}
\usepackage{graphicx}
\usepackage{balance}
\usepackage{comment}
\usepackage{amssymb,amsmath}
\usepackage{caption}
\DeclareCaptionType{copyrightbox}
\usepackage{subfigure}
\usepackage{enumerate}
\usepackage{color}
\usepackage{titling}
%\usepackage{subcaption}
\newcommand{\figref}[1]{Figure~\ref{fig:#1}}
\newcommand{\tableref}[1]{Table~\ref{tab:#1}}

\newcommand{\compactimg}{\vspace{-12pt}}

\clubpenalty=10000 
\widowpenalty=10000
%\setlength{\parindent}{0cm}



\begin{document}
%\cfoot{\thepage\ of \pageref{LastPage} }
%\rfoot{NR }

%\pagenumbering{gobble}

\begin{table}
\color{blue}
%\color{Emerald}
\begin{tabular*}{\textwidth}{l r}
\large\textbf{DIVERSITY STATEMENT} & 
\hfill \ \ \ \ \ \ \ \ \ \ \ \ \ \ \ \ \ \ \ \
\ \ \ \ \ \ \ \ \ \ \ \ \ 
\large\textbf{NIRANJINI RAJAGOPAL}\\
\hline
\end{tabular*}

\end{table}

%Over the years, I have come to recognize the factors that come in way of unequal representation of groups in STEM fields, and have take steps towards towards it through my outreach. I have taken steps towards 

%Over the years, I have come to recognize the value of diversity first-hand, and have come to recognize factors contributing to unequal representation of groups in educational institutions, and have addressed this, primarily through outreach. \\


I have found myself in settings where I feel comfortable in speaking up and sharing my ideas, and also settings where I feel reluctant or anxious. I have found a few approaches to be effective
In my experience, groups that are underrepresented in a setting are less likely to speak up and share ideas. I have found two approaches effective for addressing this.
One approach I would try

In my experience, one of the factors that contributes to unequal representation of groups in higher education is economic and financial factors. Students get interested in a domain based on their exposure to the field. One concrete step towards this is to create programs to introduce students to new domains outside through outreach. I volunteered for a short time with Agastya foundation, in editing instructor manuals for teaching middle school students Physics through experiments. The goal of the program was to inculcate curiosity and encourage students to continue their education by being interested. I volunteered with Youth for Seva back in India to raise funds for basic stationary for public schools, and organized events to visit the schools with my colleagues. I volunteered at an Oncology hospital teaching classes for kids who had discontinued their schooling. The goal of these intervensions was to keep students engaged and interested in different ways. The challenge I have faced is in sustaining these programs. How can we continue to sustain interest? \\

In the US, I volunteered with ECE outreach, where we reached out to schools in Pittsburgh, where students were unlikely to have access to engineering fields through their regular schools. Students came to CMU campus and we conducted hands-on ECE labs. This program was successful. However, it was not scalable as students would have to come to campus. As a step towards this, we started a mobile labs program where our goal was to create a truly mobile lab that could be replicated. \\

We piloted this program in a girls high school. Ran it for two years. \\
As further steps towards increasing participation of women in STEM, SWE middle school girls. Female students leading the labs, as role-models.\\

In addition to these initiatives, I have taken the effort to understand the underlying factors through a combination of my expereinces, conversations with others and through reading of research. \\
My role as a graduate teaching fellow at the Eberly Center has made a tremendous impact in my understanding of several issues through discussions of topics and research. For instance, I learnt about the stereotype threat, and realized that often intentions by instructors to be helpful, by giving extra opportunities, come across negatively by triggering the stereotype threat. The question here was, what can I actually do as an instructor? I came across literature about taking efforts for creating explicitly inclusive atmosphere, right from the design of syllabus. Subsequently in the course I TA-ed, I took the effort to meet students one-one, not only to understand their technical background but to understand their interests but to let them know that I can create time to meet them. Creating time to meet students outside of what is required is a step I have taken. In future teaching, I would like to introduce group activities. Here, I understand the challenge is in creating an overall inclusive atmosphere including peer-peer interactions. I researched this area and found that as an instructor you can still take steps. I will continue to take an approach of thoughtfully designing my courses and research groups, being mindful of these subtle phenomena what we might ignore. \\   

In research, working at the intersection of sensing systems, 

% Access to technology specifically is an economic barrier. Sometimes there aren't enough opportunities for students to be exposed 

% Growing up in India, while I had the fortune of an elite private-school education, it was far from the opportunities available for a typical student. While we cannot %While this is a complex probl
% it was far from what the I saw a wide disparity in 
% technology-oriented fields. The first is economic factors. Though school 



% I have first hand experienced the uphill battle of being a woman in STEM. 
% Personally, I have struggled with 

% Graduate school can get rough. It is unrealistic and impractical to expect alland I am aware of how it can a


% I have taken several steps. I have brought in feamles students into my research group and have mentored them. 

% What do I believe?
% We should respect each person's individual values.
% Each person should have equal opportunity.
% We should understand the underlying factors that are unconscious.
% Let each person be themselves.
% Diverse groups are productive.
% We have to be 

% As woman in STEP, I have faced several situations, where I felt I couldn't be myself,

% where I felt I . I realized that not able to be oneself .

% While I was working in India, there was a physically disabled employee in the manufacturing division who had a . Though 
% I requested the management to fund. I found stores that custom made . I realised that 

% My role as a graduate teaching fellow at the Eberly Center has made a tremendous impact in my understanding of several issues through discussions of topics and research. For isntnace, 
% I am very well aware of the stereotype threat, and have experienced it first hand as a woman in STEM. I did not realize this, unless we started reading papers about stereotype threat and along with another GTF we designed a short seminar on classroom climate, and

% I have been through tough times during my graduate school and sought out help from CMU's counceling and physhiological services. Through many conversations with graduate students, I have 

% How do my experiences make me diverse?
% Culturally, growing up in India, diversity inpersonal values

% Over the yers, I have come to recognize 

% Over the years, I have taken efforts to increase participation of women in STEM.

% I feel economic one of the factors that plays a role in 

% Mentored female students. One of my students, 

% Each student in unique. Different background\\

% Culturally, modesty 

% In my role as a Graduate Teaching Fellow at the Eberly Center at CMU, I have had several opportunities to discuss 
%  As a graduate teaching fellow at the eberly center, I designed a 30 mins seminar with two other graduate teaching fellows on "classroom climate". I researched stereotrype and steoretype threat, syllabus design and how the syllabus can accoutn for diversity, \\
%  Marginalizing centralizing, implicit and explicit.
%  (1) stereotype threat (2) syllabus (3) student-student interaction.\\

% Underrepresented minority, spent time one-one, continued to mentor, research projects, applying for PhD.\\

% Back in India, economic background. Given talks, and met students in public schools.\\

% I make myself visible to female undergraduate students.\\

% Economic status - visit schools 

% I educated myself 
% I have taken online implicit bias tests to become aware of my biases.\\
%  Taken efforts such as anonymysing student names.\\
%  Creating rubrics for grading that are clear.\\

%  Mentored female students.\\





\end{document}

